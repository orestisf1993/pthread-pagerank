%&preamble
% Save static part at preamble.tex and use command:
% xelatex -ini -shell-escape -job-name="preamble" "&xelatex preamble.tex\dump"
% to produce preamble.fmt

% Only for xelatex and lualatex. It provides an automatic and unified interface to feature-rich AAT and OpenType fonts.
% https://ctan.org/pkg/fontspec
\usepackage{fontspec}
\setmainfont{DejaVu Serif}
\renewcommand{\contentsname}{Περιεχόμενα}
\renewcommand{\listfigurename}{Λίστα Σχημάτων}
\renewcommand{\figurename}{Σχήμα}
\renewcommand{\lstlistingname}{Καταχώρηση}
\renewcommand{\lstlistlistingname}{List of \lstlistingname s}

\title{4η Εργασία στα Παράλληλα και Διανεμημένα Συστήματα\\Pagerank}
\author{Ορέστης Φλώρος-Μαλιβίτσης, 7796\\
  Χαμζάς Κωσταντίνος, 7798\\
  Τομέας Ηλεκτρονικής,\\
  Τμήμα Ηλ. Μηχανικών / Μηχανικών ΗΥ,\\
  Αριστοτέλειο Πανεπιστήμιο Θεσσαλονίκης}

\begin{document}
\maketitle
\tableofcontents
%\listoffigures
\newpage

% Δομή του Project
\chapter*{Δομή του Project} \label{project-structure}

\begin{description}
	\item[doc/] Φάκελος με τα *.tex αρχεία για την παραγωγή της αναφοράς.
	\item[doc/plot/] Φάκελος με τα αρχεία των γραφημάτων.
	\item[page-rank.pdf] Η εκφώνηση.
	\item[main.c] Ο κώδικας του προγράμματος σε c.
	\item[matlab/] Φάκελος με τα scripts σε MATLAB.
	\item[report.pdf] Αυτή η αναφορά.
	\item[run\_data.py] Script σε python3 για την παραγωγή δεδομένων και τον έλεγχό τους.
\end{description}

\chapter{Εισαγωγή}
Στη Συγκεκριμένη εργασία μας ζητείται να υλοποιήσουμε παράλληλα τον αλγόριθμο Page-rank σε ένα κατευθυνόμενο γράφο .
Ο αλγόριθμος αυτός χρησιμοποιείται για την 'μέτρηση' 
της σχετικής σημαντικότητας του κάθε κόμβου μέσα σε αυτόν τον γράφο  .
Αν θεωρήσουμε ως κατευθυνόμενο γράφο ένα κομμάτι του διαδικτύου,όπου οι κόμβο αντιστοιχούν σε ιστοσελίδες
και οι ακμές σε συνδέσεις(link) από την μία ιστοσελίδα στην άλλη τότε 
Ο αλγόριθμος είναι ο ακόλουθος:\\
  \begin{equation*}
  \bold {P^{t+1}_i = d \sum_{j \in B_i} P^t_j  \frac{a_{ij}}{ \sum_{k \in B_j} {a_{jk}} } + (1-d)E_i}
  \end{equation*}
  
  Όπου :
\begin{enumerate}
	\item $P^{t+1}_i$ η πιθανότητα να βρεθεί στον κόμβο i την χρονική στιγμή t+1
	\item $P^t_j$η πιθανότητα να βρεθεί στον κόμβο j την χρονική στιγμή t
	\item $a_{ij}$  δυαδική μεταβλητή που δείχνει την ύπαρξη σύνδεσης από τον κόμβο i στον j
	\item $B_i$ τα σύνολα των κόμβων με τους οποίους συνδέεται ο i
	\item $E_i$ η πιθανότητα ο χρήστης να βρεθεί στον κόμβο i μέσω μιας τυχαίας αναζήτησης.
	\item $\sum_{k \in B_j}$ o βαθμός συνδεσιμότητας του κόμβου j.
	\item $d$ παράγοντας που καθορίζει την σημασία του κάθε όρου της εξίσωσης  
\end{enumerate}

Μετά το πέρας ενός ικανού αριθμού βημάτων 
διάνυσμα P σταθεροποιείται αντανακλώντας τις πιθανότητες κάποιος χρήστης να βρίσκεται στην κάθε  ιστοσελίδα
κάποια μελλοντική στιγμή .(Εννοείται αν οι επιλογές του είναι μόνο οι ιστοσελίδες που έχουμε συμπεριλάβει.)
 
  Στην  εικόνα εξωφύλλου φαίνεται το  αποτέλεσμα που παράγει ένας τέτοιος αλγόριθμος. 
  

\end{document}
