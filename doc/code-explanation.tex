\chapter{Υλοποίηση σε C}

Το αρχείο main.c περιέχει την \lstinline!main()! συνάρτηση του προγράμματος
και την συνάρτηση \lstinline!calculate_pagerank()!
που είναι η κυρίως ρουτίνα για τον υπολογισμό του διανύσματος $P$.

Το αρχείο io.c περιέχει συναρτήσεις σχετικά με την επικοινωνία του προγράμματος με τον χρήστη.

\section{io.c}
Από την \lstinline!print_usage()!, οι επιλογές του προγράμματος είναι:

\begin{verbatim}
usage: ./assignment_4 [options]

	-h, --help: This help.
	-n, --nodes-file=FILENAME: File to use for input graph. Default is nodes.txt
	-t, --nthreads=NUM: Number of threads used to run pagerank.
	-f, --custom-f: Binary file to use for initial P. Default is uniform distribution.
	-e, --custom-e: Binary file to use for initial E. Default is uniform distribution.
	-s, --smart-split: Split work between threads based on workload, not nodes.
\end{verbatim}

Η συνάρτηση \lstinline!read_from_file()! ανοίγει το αρχείο που προσδιορίζεται από την επιλογή \lstinline!--nodes-file=! και
διαβάζει τον γράφο για στον οποίο θα εκτελεσθεί ο pagerank.
Γίνεται η υπόθεση ότι το αρχείο είναι απλό text file όπου τα σχόλια αρχίζουν με '\#' και υπάρχουν
2 ακέραιοι ανά γραμμή. Ο δεύτερος συμβολίζει τον node στον οποίο καταλήγει ο πρώτος.

Η συνάρτηση \lstinline!print_gen()! αποθηκεύει το αποτέλεσμα $P$ σε ένα binary file.
Η \lstinline!save_res()! αποθηκεύει πληροφορίες όπως χρόνος εκτέλεσης, αριθμός threads, αριθμός γενεών και μέγεθος.

Η συνάρτηση \lstinline!init_prob()! αρχικοποιεί τις τιμές των $P$, $E$.
Αν δεν δοθεί αρχείο από το οποίο μπορούν να διαβαστούν οι αρχικές τιμές τους, χρησιμοποιείται ομοιόμορφη κατανομή.

Η συνάρτηση \lstinline!argument_praser()! επεξεργάζεται τα ορίσματα που δίνονται στο πρόγραμμα.

\section{main.c}

Η συνάρτηση \lstinline!split_work()! χρησιμοποιείται για να αρχικοποιήσει τα ορίσματα των threads και να μοιράσει την δουλειά μεταξύ τους.
Αν είναι ενεργή η επιλογή \lstinline!--smart-split!, ο διαμοιρασμός των nodes στα threads γίνεται με βάση των αριθμό των inbound links του καθενός.
Αλλιώς, μοιράζονται ομοιόμορφα με το κάθε thread να παίρνει $\frac{N}{nthreads}$ nodes.

Η συνάρτηση \lstinline!calculate_pagerank()! πραγματοποιεί τους υπολογισμούς για το pa\-ge\-rank του $P$.
Η συνάρτηση έχει υλοποιηθεί έτσι ώστε να έχει ίδιο αποτέλεσμα με το script του MATLAB.

Στην μεταβλητή \lstinline!prob_type link_prob! υπολογίζεται η πιθανότητα από την σύνδεση μέσω κάποιου άλλου κόμβου.
Το \lstinline!node_id n_inbound[i]! είναι ο αριθμός των εισερχόμενων κόμβων στον κόμβο \lstinline!i!
ενώ ο \lstinline!node_id n_inbound[j]! ο αριθμός των εξερχόμενων από τον j.
Η εντολή \lstinline!const node_id j = L[i][x];! θέτει στην \lstinline!j! το x-οστό εισερχόμενο κόμβο στον i-οστό κόμβο.
\begin{lstlisting}[caption={υπολογισμός link\_prob}, escapechar=$]
prob_type link_prob = 0;
for (node_id x = 0; x < n_inbound[i]; x++) {
    const node_id j = L[i][x];
    link_prob += P[j] / n_outbound[j];
}
link_prob += constant_add;
\end{lstlisting}
Το \lstinline!constant_add! είναι η \hyperref[line:c_eq_0]{ποσότητα} που προστίθεται και στη συνάρτηση pagerankpow.m της MATLAB.
Υπολογίζεται σε κάθε επανάληψη από ένα thread και είναι κοινό για όλα τα threads.
Ο υπολογισμός γίνεται από την \hyperref[lst:calculate_const_add]{calculate\_const\_add()}.
Ο πίνακας \lstinline!node_id *no_outbounds! κρατάει τα στοιχεία χωρίς εξερχόμενους κόμβους.
\begin{lstlisting}[language=matlab, caption={το constant\_add στο pagerankpow.m}, escapechar=$]
if c(j) == 0
	x = x + z(j)/n;$\label{line:c_eq_0}$
else
	x(L{j}) = x(L{j}) + z(j)/c(j);
end
\end{lstlisting}

\begin{lstlisting}[caption={υπολογισμός const\_add}, escapechar=$, label={lst:calculate_const_add}]
float calculate_const_add(void) {
    // calculate the constant for links without outbound links.
    float res = 0.0f;
    for (node_id x = 0; x < size_no_out; x++) {
        const node_id j = no_outbounds[x];
        res += P[j];
    }
    res /= (prob_type) N;
    return res;
}
\end{lstlisting}

Αν χρησιμοποιείται ομοιόμορφη κατανομή για το $E$ προϋπολογίζεται η τιμή $(1-d) \cdot E$ ως:
\lstinline!const prob_type const_E = (1 - D) / (prob_type) N;!.
Αλλιώς, χρησιμοποιείται το array \lstinline!prob_type *E!.

Για τον τερματισμό των threads χρησιμοποιείται ο πίνακας \lstinline!static bool *local_terminate_flag!.
Κάθε thread ελέγχει μία \hyperref[lst:local_terminate]{συνθήκη συνέχειας} για κάθε στοιχείο του \lstinline!P!.
\begin{lstlisting}[caption={έλεγχο συνέχειας ανά thread}, escapechar=$, label={lst:local_terminate}]
P_new[i] = D * link_prob + (args->custom_E ? (1 - D) * E[i] : const_E);
if (local_terminate_flag[tid]) {
    if (fabsf(P_new[i] - P[i]) > MAX_ERROR) {
        local_terminate_flag[tid] = 0;
    }
}
\end{lstlisting}

Στο τέλος κάθε επανάληψης ένα thread αναλαμβάνει να:
\begin{itemize}
	\item κάνει \hyperref[line:swap-P-P_new]{swap} τους pointers \lstinline!P! και \lstinline!P_new! 
	\item \hyperref[line:global-check-terminate]{ελέγξει} αν οι συνθήκες συνέχειας είναι \lstinline!false! για κάθε thread
	\item \hyperref[line:call-calculate_const_add]{καλέσει} την \lstinline!calculate_const_add()!
\end{itemize}
\begin{lstlisting}[caption={swap P, P\_new και έλεγχος τερματισμού.}, escapechar=$, label={lst:gen_barrier}]
const int res = pthread_barrier_wait(&barrier);
if (res == PTHREAD_BARRIER_SERIAL_THREAD) {
    // swap P_new with P.$\label{line:swap-P-P_new}$
    prob_type *tmp;
    tmp = P;
    P = P_new;
    P_new = tmp;

    running = false;$\label{line:global-check-terminate}$
    for (unsigned int i = 0; i < nthreads; i++) {
        if (!local_terminate_flag[i]) {
            running = true;
            break;
        }
    }
    if (running) {
        constant_add = calculate_const_add();$\label{line:call-calculate_const_add}$
    }
}

pthread_barrier_wait(&barrier);
if (!running) break;
\end{lstlisting}
Τα υπόλοιπα threads περιμένουν για την επόμενη γενιά ή τερματίζονται αν \lstinline!running == false!.