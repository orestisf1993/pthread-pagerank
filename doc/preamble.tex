% Specify the type of document
\documentclass{report}
%\documentclass[draft]{report}

% packages

% https://foss.ntua.gr/wiki/index.php/%CE%95%CE%BB%CE%BB%CE%B7%CE%BD%CE%B9%CE%BA%CE%B1_%CF%83%CF%84%CE%BF_TeX/LaTeX_%CE%BC%CE%B5_%CF%84%CE%BF_XeTeX#.CE.97yphenation
% greek hyphenation package
\usepackage{xgreek}

\usepackage[textwidth=15cm]{geometry}
\usepackage{subfig}

% https://www.ctan.org/pkg/inputenc
\usepackage[utf8]{inputenc}

% Select alternative section titles
% https://www.ctan.org/pkg/titlesec
\usepackage{titlesec}

% For graphics.
% http://www.kwasan.kyoto-u.ac.jp/solarb6/usinggraphicx.pdf
% https://ctan.org/pkg/graphicx
\usepackage{graphicx}

% provides LaTeX the ability to create hyperlinks within the document.
% https://www.ctan.org/pkg/hyperref
% http://en.wikibooks.org/wiki/LaTeX/Hyperlinks#Hyperref
\usepackage[bookmarks=true,colorlinks=true]{hyperref}
% for going to the top of an image when a figure reference is clicked: http://stackoverflow.com/a/21251925/3430986%
\usepackage{caption}
% http://tex.stackexchange.com/questions/118713/is-microtype-fully-supported-now-by-xelatex-if-not-how-can-i-keep-myself-up-to
% http://tug.org/pipermail/xetex/2013-April/024263.html
\usepackage{microtype}
\usepackage{amsmath}
\usepackage{amsfonts}
\usepackage{fancyref}
%\usepackage{varwidth}

% used for figures in multicol enviroment
%\usepackage{multicol}
%\newenvironment{Figure}
%  {\par\medskip\noindent\minipage{\linewidth}}
%  {\endminipage\par\medskip}

\titleformat{\chapter}[hang]{\bf\huge}{\thechapter}{2pc}{}
% plain, headings or empty. Headins prints the current chapter heading + page number
%\pagestyle{headings}
% fancier headings with these package
\usepackage{fancyhdr}
\pagestyle{fancy}
% with this we ensure that the chapter and section
% headings are in lowercase.
\renewcommand{\chaptermark}[1]{%
  \markboth{#1}{}}
\renewcommand{\sectionmark}[1]{%
  \markright{\thesection\ #1}}
\fancyhf{} % delete current header and footer
\fancyhead[LE,RO]{\bfseries\thepage}
\fancyhead[LO]{\bfseries\rightmark}
\fancyhead[RE]{\bfseries\leftmark}
\renewcommand{\headrulewidth}{0.5pt}
\renewcommand{\footrulewidth}{0pt}
\addtolength{\headheight}{0.5pt} % space for the rule
\fancypagestyle{plain}{%
  \fancyhead{} % get rid of headers on plain pages
  \renewcommand{\headrulewidth}{0pt} % and the line
}

% http://ctan.org/pkg/enumerate
\usepackage{enumerate}
% http://tex.stackexchange.com/a/56953/78791
\newcommand{\argmin}{\operatornamewithlimits{argmin}}
% http://tex.stackexchange.com/a/107190/78791
\newcommand{\norm}[1]{\left\lVert#1\right\rVert}
\newcommand{\abs}[1]{\left\lvert#1\right\rvert}
\newcommand{\overbar}[1]{\mkern 1.5mu\overline{\mkern-1.5mu#1\mkern-1.5mu}\mkern 1.5mu}
% bad fonts for implies?
\renewcommand{\implies}{=\!\Rightarrow}
\usepackage{extarrows}

% Used for pgf plots. Taken from https://github.com/matlab2tikz/matlab2tikz
% also see http://www.howtotex.com/packages/beautiful-matlab-figures-in-latex/
%\usepackage{pgfplots}
%\pgfplotsset{compat=newest}
%\pgfplotsset{plot coordinates/math parser=false}
%\newlength\figureheight
%\newlength\figurewidth

% for code with higlighting
\usepackage{listings}
\usepackage{color}
\definecolor{mygreen}{rgb}{0,0.6,0}
\definecolor{mygray}{rgb}{0.5,0.5,0.5}
\definecolor{mymauve}{rgb}{0.58,0,0.82}
\lstset{ %
  backgroundcolor=\color{white},   % choose the background color; you must add \usepackage{color} or \usepackage{xcolor}
  basicstyle=\footnotesize,        % the size of the fonts that are used for the code
  breakatwhitespace=false,         % sets if automatic breaks should only happen at whitespace
  breaklines=true,                 % sets automatic line breaking
  captionpos=b,                    % sets the caption-position to bottom
  commentstyle=\color{mygreen},    % comment style
  escapeinside={\%*}{*)},          % if you want to add LaTeX within your code
  extendedchars=true,              % lets you use non-ASCII characters; for 8-bits encodings only, does not work with UTF-8
  frame=single,                    % adds a frame around the code
  keepspaces=true,                 % keeps spaces in text, useful for keeping indentation of code (possibly needs columns=flexible)
  keywordstyle=\color{blue},       % keyword style
  language=c,                 % the language of the code
  numbers=left,                    % where to put the line-numbers; possible values are (none, left, right)
  numbersep=5pt,                   % how far the line-numbers are from the code
  numberstyle=\tiny\color{mygray}, % the style that is used for the line-numbers
  rulecolor=\color{black},         % if not set, the frame-color may be changed on line-breaks within not-black text (e.g. comments (green here))
  showspaces=false,                % show spaces everywhere adding particular underscores; it overrides 'showstringspaces'
  showstringspaces=false,          % underline spaces within strings only
  showtabs=false,                  % show tabs within strings adding particular underscores
  stepnumber=1,                    % the step between two line-numbers. If it's 1, each line will be numbered
  stringstyle=\color{mymauve},     % string literal style
  tabsize=4,                       % sets default tabsize to 2 spaces
  title=\lstname                   % show the filename of files included with \lstinputlisting; also try caption instead of title
}
%\endofdump
