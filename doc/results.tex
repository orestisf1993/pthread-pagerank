\chapter{Αποτελέσματα}
Σε αυτό το κομμάτι παρουσιάζονται όλα τα πειραματικά αποτελέσματα 
\section{Έλεγχος ορθότητας}
Εδώ φαίνονται τα τελικά αποτελέσματα (rankings ) που προέκυψαν από την εφαρμογή του αλγορίθμου σε 2 διαφορετικούς γράφους .
Το 1 πρώτο προέκυψε με την χρήση της συνάρτηση surfer.m στη σελίδα stackoveflow.com
 
O γράφος που δημιουργήθηκε από το stackoveflow είναι ο ακόλουθος:
 \begin{figure}[h!]
 \centerline{\includegraphics[width=0.5\textwidth]{stackGraph.pdf}}
 \caption{Ο γράφος συνδέσεων}
 \end{figure}
 
 Παρατηρούμε το εξής το και το εξής
 \newpage
 Τα αποτελέσματα που πήραμε για το ranking είναι τα εξής για τόσο από τη c όσο και από το matlab
 \begin{figure}[h!]
 \centerline{\includegraphics[width=0.85\textwidth]{stackres.pdf}}
 \end{figure}
 
  Βλέπουμε ότι τα αποτελέσματα μεταξύ των 2 υλοποιήσεων  ταυτίζονται σε πολύ μεγάλο βαθμό .
  Επίσης σε σχέση  με Σχήμα 3.1  μπορούμε να επιβεβαιώσουμε  εποπτικά την ορθότητα των αποτελέσματά μας
  παρατηρώντας ότι οι μεγαλύτερες πιθανότητες εμφανίζονται  στου κόμβους 40 και 90 οι οποίοι  σύμφωνα με το Σχήμα 3.1 έχουν
  τα περισσότερα links προς αυτούς (κάθετος άξονας ).Επίσης ο αριθμός των επαναλήψεων που χρειάστηκε για να συγκλίνει ο αριθμός σε σταθερή τιμή ήταν 40 .Ο κάνονας που θέσαμε ήταν να μην υπάρχει μεταβολή μεγαλύτερη του $10^{-6}$ στην επόμενη επανάληψη για 
  οποιονδήποτε κόμβο.Η μεγάλη συνδεσιμότητα του γράφου δικαιολογεί το σχετικά μεγάλο αυτό νούμερο καθώς οι μεταβολές θα ήταν πολύ πυκνές και συχνά αλληλοαναιρούμενες 
 \newpage

 Στο δεύτερο χρησιμοποιήθηκε αυτός ο γράφος όπου προέρχεται από αυτό το
 \href {http://snap.stanford.edu/data/soc-Slashdot0811.html}{link}
και περιέχει ένα δίκτυο από το   \url{ http://slashdot.org/}
 
  O γράφος του slashdot είναι ο ακόλουθος:
 \begin{figure}[h!]
 \centering
 \subfloat[]{\includegraphics[width=0.5\textwidth]{slashGraph.pdf}}
 \subfloat[]{\includegraphics[width=0.5\textwidth]{slashGraphDet.pdf}}
 \caption{O γράφος συνδέσεων με μία λεπτομέρεια στα αριστερά}
 \end{figure}

 Ο γράφος δεν είναι τόσο πυκνός όσο φαίνεται δίπλα του βρίσκεται μία λεπ 1.5130e-04τομέρεια στην πάνω αριστερά γωνία όπου διαφένεται ότι είναι 
 αρκετά αραιός έχει πυκνότητα τόση 1.5130e-04 15 συνδέσεις ανά  100.000 κόμβους 
\newpage 
Το ranking που προκύπτει για αυτό το δίκτυο είναι το εξής:
   \begin{figure}[h!]
\centerline{ \includegraphics[width=0.85\textwidth]{slashdot.pdf}}
\end{figure}
 Επιβεβαιώνουμε την ορθότητα των αποτελεσμάτων μας με με το ίδιο σκεπτικό όπως στην προηγούμενη περίπτωση.
 Βλέπουμε λοιπόν ότι η παράλληλη υλοποίηση δουλεύει σωστά και για πολύ μεγαλύτερο αριθμό κόμβων.Τέλος 
 πρέπει να σημειώσουμε   ότι τα βήματα που χρειάστηκαν για την σύγκλιση ήταν μόλις 19!.Θα περίμενε κανείς ότι ενας
 μεγαλύτερος γράφος με τόσο μεγάλη ασυμμετρία ότι θα ήθελε περισσότερα βήματα για να συγκλίνει σε σχέση με το γράφο του stackoveflow .Το γεγονός όμως ότι είναι πολύ πιο αραιός τον κάνει να συγκλίνει στο τελικό διάνυσμα Πολύ πιο γρήγορα
 
 
 
\section{Χρόνος εκτέλεσης}
\section{Σύγκλιση γενεών}